\documentclass[12pt]{report}

% ------------------------------------------------
% Packages
% ------------------------------------------------
\usepackage{graphicx}     % for figures
\usepackage{amsmath, amssymb} % for math
\usepackage{hyperref}     % for hyperlinks
\usepackage{natbib}       % for references
\usepackage{geometry}     % adjust margins
\geometry{margin=1in}

% ------------------------------------------------
% Title Page
% ------------------------------------------------
\begin{document}

\title{Minimal Explanations for Unsatisfiability in Mission-Time Linear Temporal Logic (MLTL)}
\author{Your Name}
\date{\today}
\maketitle

\tableofcontents
\listoffigures
\listoftables

% ------------------------------------------------
% Chapters
% ------------------------------------------------

\chapter{Introduction}
\begin{itemize}
  \item Motivation: challenges in debugging MLTL specifications.
  \item Problem statement: unsatisfiable specifications are difficult to interpret.
  \item Thesis goals: create a tool that extracts unsat cores, adapts them to runtime verification, and presents minimal explanations.
  \item Contributions:
    \begin{enumerate}
      \item Tool: MLTL Unsat Core Tool.
      \item Method: adaptation of unsat-core extraction to minimal variable+timestep explanations.
      \item HCI: visualization + user study on interpretability.
    \end{enumerate}
  \item Thesis structure overview.
\end{itemize}

\chapter{Background and Related Work}
\begin{itemize}
  \item Mission-Time Linear Temporal Logic (MLTL).
  \item Unsatisfiable cores: SAT/SMT methods (QuickXplain, Z3, etc.).
  \item Runtime verification: goals and challenges.
  \item Visualization and HCI in formal methods tools.
\end{itemize}

\chapter{System Design and Implementation}
\begin{itemize}
  \item Tool architecture: backend solver + frontend (React).
  \item Input format: traces and specifications.
  \item Workflow: trace $\rightarrow$ solver $\rightarrow$ unsat core.
  \item Example run with toy problem.
\end{itemize}

\chapter{Methodology: Minimal Explanations}
\begin{itemize}
  \item Problem framing: minimal variables and timesteps.
  \item Adaptation of existing algorithms to runtime verification.
  \item Pseudocode for explanation extraction.
  \item Example walk-through: large trace with conflict at $t=51$.
\end{itemize}

\chapter{Visualization and Human-Centered Design}
\begin{itemize}
  \item Design goals: reduce cognitive overload, highlight key variables.
  \item Interface features: variable highlighting, timestep focus.
  \item Rationale for design choices.
  \item Screenshots/mockups.
\end{itemize}

\chapter{Evaluation}
\begin{itemize}
  \item Study design: participants, tasks, measures.
  \item Pilot study results and refinements.
  \item Main study: results (quantitative and qualitative).
  \item Analysis of tool effectiveness.
\end{itemize}

\chapter{Discussion}
\begin{itemize}
  \item Summary of findings.
  \item Lessons for runtime verification tools.
  \item Limitations of current approach.
\end{itemize}

\chapter{Conclusion and Future Work}
\begin{itemize}
  \item Summary of contributions.
  \item Implications for verification and HCI.
  \item Directions for extending this work (scalability, industrial applications).
\end{itemize}

% ------------------------------------------------
% References
% ------------------------------------------------
\bibliographystyle{plainnat}
\bibliography{ref}

\end{document}
